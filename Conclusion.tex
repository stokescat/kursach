\section{Заключение}
В ходе выполнения курсовой работы разработан метод фаззинг-тестирования драйверов файловых систем в UEFI-окружении. Основные результаты включают:
\begin{enumerate}
	\item Доработку пользовательского окружения подсистемой событий UEFI Event.
	\item Реализацию шести тестовых сценариев, охватывающих синхронные и асинхронные операции с файлами и каталогами через стандартизированный интерфейс \VarName{EFI\_FILE\_PROTOCOL}.
	\item Автоматизированную генерацию начальных корпусов из образов тестируемых файловых систем.
	\item Обнаружение и исправление ошибок в драйверах.
	\item Получение метрик покрытия кода для драйверов FAT и ext4.
\end{enumerate}

В продолжении этой работы планируется:
\begin{itemize}
	\item Продолжение фаззинг-тестирования драйверов FAT, ext4 и NTFS с исправлением выявленных ошибок.
	\item Оптимизация используемых сценариев, подбор оптимальных параметров, чтобы ускорить тестирование.
	\item Исследовать сценарий, при котором синхронные операции не дожидаются завершения асинхронных.
	\item Исследование ресурсоемкости метода. Даже несмотря на использование различных типов аллокаторов экспоненциально растет потребление памяти, что не дает запустить фаззер на долгое время. Возможно требуется провести исследование потребления памяти у используемого окружения.
\end{itemize} 