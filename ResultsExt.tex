\section{Результаты тестирования драйвера ext4}

\subsection{Подготовка к тестированию}
Для тестирования драйвера файловой системы ext4 были разработаны bash-скрипты (\textit{Test1.sh}-\textit{Test10.sh}), которые автоматизируют создание тестовых образов для начального корпуса. Подробное описание каждого скрипта приведено в таблице \ref{ext:tab:test_detail}. Основные характеристики генерируемых образов:
\begin{itemize}
	\item Многоуровневые вложенные директории (до 8 уровней)
	\item Различные типы ссылок (симлинки, жесткие ссылки)
	\item Специальные случаи: ссылки на себя, родительские и корневые директории
	\item Ссылки за пределы образа
	\item Жесткие ссылки на один inode
	\item Отдельный тест создает длинные имена (200-255 символов)
	\item Заполнение образа до заданного порога свободного места при помощи папки FILLSPACE
	\item Различные размеры файов (от 1 Кб до 2 Мб) 
	\item Создание некорректных симлинков
	\item Все скрипты создают образ в 64 Мб
\end{itemize} 
\begin{table}[htbp]
	\renewcommand{\arraystretch}{1.5}
	\centering
	\begin{tabular}{|c|p{10cm}|}
		\hline
		\textbf{Скрипт} & \textbf{Описание} \\
		\hline
		\textit{Test1.sh} & Базовая структура: 4 корневые папки (по 10 файлов), 3 вложенные цепочки (глубина 3), 5 файлов по 2 МБ в корне\\
        \hline
        \textit{Test2.sh} & Структура Test1 + симлинки в корне на поддиректории уровня 2\\
        \hline
        \textit{Test3.sh} & 1 папка (10 файлов) + симлинк на себя внутри папки + симлинк на папку в корне + 5 файлов по 2 МБ\\
        \hline
        \textit{Test4.sh} & Структура Test3 + симлинк на корневую директорию внутри папки \\
        \hline
        \textit{Test5.sh} & Структура Test4 + симлинк на внешнюю папку за пределами образа \\
        \hline
        \textit{Test6.sh} & Глубокая вложенность (8 уровней) + файлы на каждом уровне + симлинки на предыдущий уровень и корень (за пределы образа) \\
        \hline
        \textit{Test8.sh} & Создание: 50 папок (по 100 файлов по 1K) + симлинки между папками $(i → i+10)$, где $i=1...10$. Заполнение FILLSPACE отсутствует\\
        \hline
        \textit{Test9.sh} & Жесткие ссылки: 1 исходный файл + 15 жестких ссылок (10 в корне, 5 в подпапке) + 1 несвязанный файл\\
        \hline
        \textit{Test10.sh} & Длинные имена (200-255 символов): директория + 5 файлов + симлинк + файл в корне\\
        \hline
	\end{tabular}
	\caption{Подробные характеристики скриптов для создания образов}
	\label{ext:tab:test_detail}
\end{table}

\newpage
\subsection{Исправленные ошибки}
\begin{enumerate}
	\item \textbf{Не следование спецификации}
	
	\textbf{Файл:} \FName{Partition.c}
	
	\textbf{Проблема:} Драйвер не поставляет функцию \FunName{Flush}, хотя по спецификации она обязана присутствовать \cite{UEFISpec}
	
	\textbf{Решение:} Так как драйвер ReadOnly, то реализована заглушка  \FunName{\_Ext4Flush} (см. листинг \ref{ext:listingA}) и заполняется поле \VarName{Flush} файлового протокола в функции \FunName{Ext4SetupFile} (см. листинг \ref{ext:listingB})
	\lstinputlisting[
	caption={Файл \FName{Partition.c}: функция-заглушка \FunName{\_Ext4Flush}},
	label={ext:listingA}
	]{extlistings/listingA.c} 
	\lstinputlisting[
	caption={Файл \FName{Partition.c}, кусок кода функции \FunName{Ext4Setup}: установка заглушки},
	label={ext:listingB}
	]{extlistings/listingB.c} 
	
	\item \textbf{Раскрытие ASSERT} в функции \FunName{Ext4ReadFile}
	
	\textbf{Файл:} \FName{File.c}
	
	\textbf{Проблема:} Срабатывание \FunName{ASSERT}
	
	\textbf{Решение:} Заменить \FunName{ASSERT} на явную обработку ошибки (листинг \ref{ext:listingC})
	 \lstinputlisting[
	 caption={Файл \FName{File.c}, кусок кода функции \FunName{Ext4ReadFile}: явная обработка \FunName{ASSERT}},
	 label={ext:listingC}
	 ]{extlistings/listingC.c} 
\end{enumerate}

Итоговое количество обнаруженных и исправленных на данный момент ошибок сведено в таблице \ref{ext:tab:sumerror}.
\begin{table}[htbp]
	\renewcommand{\arraystretch}{1.5}
	\centering
	\begin{tabular}{|c|c|}
		\hline
		\textbf{Тип} & \textbf{Количество} \\
		\hline
		Не следование спецификации & 1 \\
		\hline
		\ErrorName{Assert} & 1 \\
		\hline
	\end{tabular}
	\caption{Тип и количество исправленных ошибок в драйвере ext4}
	\label{ext:tab:sumerror}
\end{table}

\newpage
\subsection{Покрытие кода}
Результирующее покрытие кода драйвера ext4 сведено в таблице \ref{ext:tab:coverage}
\begin{table}[ht]
	\centering
	\begin{tabular}{|c|c|c|c|}
		\hline
		\textbf{Filename} & \textbf{Line} & \textbf{Branch} & \textbf{Function} \\
		\hline
\FName{BlockGroup.c} & 75.0\% & 54.5\% & 83.3\% \\
\FName{BlockMap.c} & 0.0\% & 0.0\% & 0.0\% \\
\FName{Directory.c} & 87.3\% & 80.4\% & 100.0\% \\
\FName{DiskUtil.c} & 73.1\% & 50.0\% & 100.0\% \\
\FName{Ext4Dxe.c} & 0.0\% & 0.0\% & 0.0\% \\
\FName{Extents.c} & 48.5\% & 36.5\% & 73.3\% \\
\FName{File.c} & 66.8\% & 62.9\% & 85.0\% \\
\FName{Inode.c} & 88.5\% & 59.5\% & 100.0\% \\
\FName{Partition.c} & 90.0\% & 66.7\% & 100.0\% \\
\FName{Superblock.c} & 66.0\% & 55.7\% & 83.3\% \\
\FName{Symlink.c} & 66.7\% & 59.4\% & 100.0\% \\
		\hline
	\end{tabular}
	\caption{Покрытие кода драйвера ext4}
	\label{ext:tab:coverage}
\end{table}
