\section{Результаты тестирования драйвера FAT}

\subsection{Подготовка к тестированию}
Для запуска фаззинг-тестирования драйвера файловой системы FAT необходим начальный корпус из тестовых образов. Их генерация автоматизирована с помощью набора bash-скриптов (\textit{Test1.sh}-\textit{Test7.sh}). Скрипты создают образы, моделирующие разнообразные сценарии использования FAT.  Общая характеристика генерируемых образов:
\begin{itemize}
	\item Создаются образы FAT12, FAT16 и FAT32
	\item Различные размеры образов: от 8 Мб (маленькие образы для FAT12/FAT16) до 64 Мб (стандартный размер для FAT32)
	\item Используются разные размеры кластеров (512 байт и 2 Кб)
	\item Создается заданное количество папок (от 1 до 5) с файлами фиксированного (8 Кб) размера
	\item В корне размещаются файлы значительно большого размера (512 Кб, 2 Мб)
	\item Создаются цепочки вложенных папок (глубина от 2 до 3 уровней) с файлами на конечном уровне вложенности
	\item Большинство скриптов создает специальную папку FILLSPACE, которая заполняется по остаточному принципу до достижения заданного порога свободного места (1 Мб, 2 Мб, 8Мб) для имитации работы с почти заполненными образами
	\item Имена файлов и папок (длиной 32 символа) генерируются случайно из буквенно-цифрового набора. Скрипт \textit{Test7.sh} дополнительно включает имена с пробелами и специальными символами, проверяя корректность их обработки.
	\item Создаваемые файлы заполняются случайными данными из Linux-устройства \textit{/dev/urandom} 
\end{itemize} 
Такой подход к формированию начального корпуса позволяет эффективно тестировать базовые функции драйвера (чтение, запись, поиск, открытие/закрытие файлов и каталогов) в экстремальных и разнообразных условиях. Подробные параметры каждого скрипта сведены в таблице \ref{fat:tab:test_detail}.
\begin{table}[htbp]
	\renewcommand{\arraystretch}{1.5}
	\centering
	\begin{tabular}{|c|c|c|p{10cm}|}
		\hline
		\textbf{Скрипт} & \textbf{Размер} & \textbf{FAT} & \textbf{Описание} \\
		\hline
		\textit{Test1.sh} & 64 Мб & 32 &   4 корневые папки (по 10 файлов 8 Кб), 3 вложенные цепочки (глубина 3) с 10 файлами по 8 Кб на конце, 5 файлов в корне по  2Мб. FILLSPACE до 8 Мб. Размер кластера 512 байт \\
		\hline
		\textit{Test2.sh} & 32 Мб & 16 &   Структура идентична Test1 \\
		\hline
		\textit{Test3.sh} & 64 Мб & 16 &  Структура идентична Test2, но размер кластера увеличен до 2 Кб\\
		\hline
		\textit{Test4.sh} & 8 Мб & 12 &  2 корневые папки (по 5 файлов 8Кб), 1 вложенная цепочка (глубина 3) с 5 файлами по 8 Кб на конце,  3 файла в корне по 4 Кб. FILLSPACE до 2 Мб. Размер кластера 2 Кб \\
		\hline
		\textit{Test5.sh} & 64 Мб & 32 &  5 корневых папок (по 15 файлов 8 Кб), 4 вложенные цепочки (глубина 3) с 15 файлами по 8 Кб на конце, 2 файла  в корне по 2 МБ. Без заполнения FILLSPACE.  Размер кластера 2 Кб\\
		\hline
		\textit{Test6.sh} & 8 Мб & 16 &  Минималистичная структура: 1 корневая папка (3 файла), 1 вложенная цепочка (глубина 2, 3 файла), 2 файла по 512 КБ.  FILLSPACE до 1 Мб. Размер кластера 512 байт\\
		\hline
		\textit{Test7.sh} & 64 Мб & 32 & 3 корневые папки (по 5 файлов), 4 файла по 1 МБ, 2 вложенные цепочки (глубина 3, 3 файла). Имена с пробелами и спецсимволами. Без FILLSPACE. Размер кластера 512 байт  \\
		\hline
		
	\end{tabular}
	\caption{Подробные характеристики скриптов для создания образов}
	\label{fat:tab:test_detail}
\end{table}

Чтобы тестировать асинхронный функционал драйвера FAT, было отключено кеширование при вызове асинхронных функций, чтобы не ждать переполнение кеша, иначе операции все равно будут выполнены синхронно. Для этого в файле \FName{ReadWrite.c} драйвера были исправлены функции \FunName{FatReadEx} и \FunName{FatWriteEx} (см. листинг \ref{fat:listingA}): в вызов функции \FunName{FatIFileAccess} передаем флаги \FlagName{ReadDisk}/\FlagName{WriteDisk} вместо \FlagName{ReadData}/\FlagName{WriteData} 
\lstinputlisting[
caption={Файл \FName{ReadWrite.c}: изменения в функциях \FunName{FatReadEx} и \FunName{FatWriteEx}},
label={fat:listingA}
]{fatlistings/listingA.c} 

\subsection{Исправленные ошибки}

В процессе фаззинг-тестирования были обнаружены и исправлены следующие критические ошибки.

\begin{enumerate}
	\item \textbf{Неопределенное поведение (UB)} в функциях \FunName{FatGetDirEntInfo} и \FunName{FatOpenDirEnt} 
	
	\textbf{Файл} \FName{DirectoryManage.c}
	
	\textbf{Проблема:} При сборке номера кластера \VarName{Cluster} из 16-битных компонент сдвиг влево на 16 бит вызвал неявное приведение к типу \TypeName{int}, что привело к \ErrorName{UB} при переполнении \cite{ISO_C17}
	
	\textbf{Исправление:} Явное приведение к \TypeName{UINTN} (см. листинги \ref{fat:listingB} и \ref{fat:listingC})
	\lstinputlisting[
	caption={Файл \FName{DirectoryManage.c}, кусок кода функции \FunName{FatGetDirEntInfo}:  BEFORE - до исправления, AFTER - после исправления},
	label={fat:listingB}
	]{fatlistings/listingB.c} 
	\lstinputlisting[
	caption={Файл \FName{DirectoryManage.c}, кусок кода функции \FunName{FatOpenDirEnt}:  BEFORE - до исправления, AFTER - после исправления},
	label={fat:listingC}
	]{fatlistings/listingC.c} 
	
	\item \textbf{Утечка памяти (MemoryLeak) и взаимоблокировка (DeadLock)} в функции \FunName{FatIFileAccess}
	
	\textbf{Файл:} \FName{ReadWrite.c}
	
	\textbf{MemoryLeak:} При ошибке в \FunName{FatGrowEof} не освобождается объект асинхронной задачи
	
	\textbf{DeadLock:} Блокировка тома не снималась перед вызовом \FunName{FatQueueTask}, что может вызвать взаимоблокировку при обработке связанного с задачей сигнала
	
	\textbf{Исправление:}
	\begin{itemize}
		\item Добавлено освобождение задачи при ошибках (строки 9-11 листинга \ref{fat:listingD})
		\item Снятие блокировки тома перед \FunName{FatQueueTask} и при ошибках (строки 13 и 19 листинга  \ref{fat:listingD})
	\end{itemize}
	\lstinputlisting[
	caption={Файл \FName{ReadWrite.c}, кусок кода функции \FunName{FatIFileAccess}:  комментариями отмечены типы исправленных ошибок},
	label={fat:listingD}
	]{fatlistings/listingD.c} 
	
	\item \textbf{Доступ после освобождения (UseAfterFree)} в \FunName{FatQueueTask}
	
	\textbf{Файл:} \FName{Misc.c}
	
	\textbf{Проблема:} Итерация по списку подзадач без блокировок приводила к доступу к освобождённой памяти, если элементы удалялись в обработчике событий
	
	\textbf{Исправление:} Добавлены блокировки \VarName{FatTaskLock} во время итераций по списку (см. листинг \ref{fat:listingE})
	\lstinputlisting[
	caption={Файл \FName{Misc.c}:  переписанная функция \FunName{FatQueueTask}, многоточиями отмечены похожие с предыдущей версией места},
	label={fat:listingE}
	]{fatlistings/listingE.c} 
	
\end{enumerate}

\newpage
Итоговое количество обнаруженных и исправленных на данный момент ошибок сведено в таблице \ref{fat:tab:sumerror}.
\begin{table}[htbp]
	\renewcommand{\arraystretch}{1.5}
	\centering
	\begin{tabular}{|c|c|}
		\hline
		\textbf{Тип} & \textbf{Количество} \\
		\hline
		\ErrorName{UB} & 2 \\
		\hline
		\ErrorName{MemoryLeak} & 1 \\
		\hline
		\ErrorName{DeadLock} & 1 \\
		\hline
		\ErrorName{UseAfterFree} & 1 \\
		\hline
	\end{tabular}
	\caption{Тип и количество исправленных ошибок в драйвере FAT}
	\label{fat:tab:sumerror}
\end{table}

\subsection{Покрытие кода}
Результирующее покрытие кода драйвера FAT сведено в таблице \ref{fat:tab:coverage}
\begin{table}[ht]
	\centering
	\begin{tabular}{|c|c|c|c|}
		\hline
		\textbf{Filename} & \textbf{Line} & \textbf{Branch} & \textbf{Function} \\
		\hline
		\FName{Delete.c}&  58.3\% &  32.4\% &  100.0\% \\
		\FName{DirectoryCache.c} &  100.0\% &  70.6\% &  100.0\% \\
		\FName{DirectoryManage.c} &  76.7\% &  62.9\% &  80.8\% \\
		\FName{DiskCache.c} &  80.1\% &  76.6\% &  85.7\% \\
		\FName{FileName.c} &  89.1\% &  86.7\% &  100.0\% \\
		\FName{FileSpace.c} &  88.0\% &  79.0\% &  91.7\% \\
		\FName{Flush.c} &  86.6\% &  70.2\% &  100.0\% \\
		\FName{Hash.c} &  100.0\% &  100.0\% &  100.0\% \\
		\FName{Info.c} &  37.6\% &  30.4\% &  55.6\% \\
		\FName{Init.c} &  73.3\% &  51.1\% &  66.7\% \\
		\FName{Misc.c} &  87.2\% &  62.8\% &  94.1\% \\
		\FName{Open.c} &  85.4\% &  72.9\% &  100.0\% \\
		\FName{OpenVolume.c} &  84.6\% &  50.0\% &  100.0\% \\
		\FName{ReadWrite.c} &  87.3\% &  69.4\% &  100.0\% \\
		\FName{UnicodeCollation.c} &  47.4\% &  0.0\% &  71.4\% \\
		\hline
	\end{tabular}
	\caption{Покрытие кода драйвера FAT}
	\label{fat:tab:coverage}
\end{table}