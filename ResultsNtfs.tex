\section{Результаты тестирования драйвера EFI NTFS}

\subsection{Подготовка к тестированию}
Аналогично FAT и ext4 был разработан набор bash-скриптов (\textit{Test1.sh}-\textit{Test8.sh}), создающих специализированные тестовые образы файловой системы NTFS, которые составят начальный корпус для фаззинг-тестирования. Эти скрипты охватывают ключевые особенности NTFS, включая поддержку различных размеров кластеров, работу с метаданными, обработку сложных структур ссылок и экстремальных случаев. Подробные характеристики приведены в таблице \ref{ntfs:tab:test_detail}. Основные характеристики генерируемых образов:
\begin{itemize}
	\item Многоуровневые вложенные директории (до 4 уровней).
	\item Различные типы ссылок (символические и жесткие).
	\item Специальные случаи (ссылки на себя, корневые директории, за пределы образа).
	\item Различные размеры кластеров.
	\item Различные размеры образов (8 Мб - 32 Мб).
	\item Аналогичное заполнение до указанного порога (FILLSPACE, только Test1.sh).
	\item Генерация как стандартных, так и длинных имен со специальными символами.
	\item Жесткие ссылки на один файл.
\end{itemize} 

\begin{table}[htbp]
	\renewcommand{\arraystretch}{1.5}
	\centering
	\begin{tabular}{|c|c|p{10cm}|}
		\hline
		\textbf{Скрипт} & \textbf{Размер} & \textbf{Описание} \\
		\hline
		\textit{Test1.sh} & 16 Мб & Базовая структура: 4 корневые папки (по 10 файлов), 3 вложенные цепочки (глубина 3), 3 файла по 2 Мб в корне. FILLSPACE до 2 Мб. \\
		\hline
		\textit{Test2.sh} & 32 Мб & Нестандартный размер кластера (8 Кб). 5 корневых папок (по 15 файлов), 4 вложенные цепочки (глубина 4), 10 файлов по 1 Мб.\\
		\hline
		\textit{Test3.sh} & 32 Мб &Длинные имена (200 символов) со спецсимволами. 20 папок (по 200 файлов по 1 Кб) + символические ссылки на себя и из корня. \\
		\hline
		\textit{Test4.sh} & 8 Мб & Маленький образ. 20 папок (по 10 файлов по 1 Кб) + символические ссылки на себя в каждой папке и из корня на папки.\\
		\hline
		\textit{Test5.sh} & 16 Мб & Жесткие ссылки: 1 базовый файл (2 Мб) + 10 ссылок в корне + 5 ссылок в подпапке.\\
		\hline
		\textit{Test6.sh} & 16 Мб & Массовое создание файлов: 10 папок × 500 файлов (всего 5000). \\
		\hline
		\textit{Test7.sh} & 16 Мб & Папки с символическими ссылками: 5 папок (по 10 файлов), в каждой символические ссылки на себя и на корневую директорию.\\
		\hline
		\textit{Test8.sh} & 16 Мб & Символическая ссылка на внешнюю директорию: 5 папок (по 10 файлов), в каждой символические ссылки на себя, корень и внешнюю папку за пределами образа.\\
		\hline
	\end{tabular}
		\caption{Подробные характеристики скриптов для создания образов.}
	\label{ntfs:tab:test_detail}
\end{table}

\newpage
\subsection{Ошибки, обнаруженные при тестировании}
\begin{enumerate}
	\item \textbf{Не следование спецификации.}
	
	\textbf{Файл:} \FName{NTFS.c}.
	
	\textbf{Проблема:} В функции \FunName{NTFSStart} указывается некорректная версия файлового протокола (см. листинг \ref{ntfs:listingA}), вторая версия файлового протокола требует наличия асинхронных функций, которые драйвер не поставляет \cite{UEFISpec}.
    
    \textbf{Решение:} Указать правильный номер версии.
	\lstinputlisting[
	caption={Файл \FName{NTFS.c}, кусок кода функции \FunName{NTFSStart}: некорректная версия файлового протокола},
	label={ntfs:listingA}
	]{ntfslistings/listingA.c} 
	
	\item \textbf{Не возвращается признак конца каталога}.
	
	\textbf{Файл:} \FName{Open.c}.
	
	\textbf{Проблема:} Функция \FunName{FileRead} не возвращает признак конца каталога, как того требует спецификация \cite{UEFISpec}.
	
	\textbf{Текущий статус:} Не решена.
\end{enumerate}

\subsection{Покрытие кода}
На момент сдачи работы измерение покрытия кода для драйвера EFI NTFS \textbf{невозможно} из-за критической ошибки, приводящей к зависанию тестов при работе с каталогами. Для получения валиндных метрик необходимо исправить эту ошибку. 