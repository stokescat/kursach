\documentclass{HSECourseW}

\usepackage{graphicx}


\renewcommand{\TitleFaculty}{Факультет компьютерных наук}
\renewcommand{\TitleProgram}{<<Системное программирование>>}
\renewcommand{\TitleDepartment}{программной инженерии}
\renewcommand{\TitleTheme}{Разработка метода фаззинг-тестирования драйверов файловых систем для UEFI-окружения}
\renewcommand{\TitleGroupNum}{МСТПР241}
\renewcommand{\TitleAuthor}{Набережнев Павел Александрович}
\renewcommand{\TitleSupervisor}{доцент, к.ф.-м.н., А. В. Хорошилов}
\renewcommand{\TitleConsultant}{  В. Ю. Чепцов  }
\renewcommand{\TitleCity}{Москва}
\renewcommand{\TitleYear}{2025}

\newcommand{\FName}[1]{\textit{#1}}
\newcommand{\FunName}[1]{\textit{#1()}}
\newcommand{\FlagName}[1]{\textit{#1}}
\newcommand{\ErrorName}[1]{\textit{#1}}
\newcommand{\VarName}[1]{\textit{#1}}
\newcommand{\TypeName}[1]{\textbf{#1}}
\newcommand{\GitName}[1]{\textbf{#1}}

\begin{document}
	
\titlepage

\begin{abstractpage}
	В работе разработан и реализован метод фаззинг-тестирования драйверов файловых систем, функционирующих в среде UEFI. Предложенный подход включает эмуляцию ключевых сервисов UEFI (дисковых устройств и подсистемы событий) в пользовательском пространстве и набор специализированных тестовых сценариев, охватывающих как синхронные, так и асинхронные операции через стандартный интерфейс EFI\_FILE\_PROTOCOL. Метод позволяет автоматизированно выявлять критические ошибки (неопределенное поведение, утечки памяти, взаимоблокировки) в драйверах до их развертывания. Результаты тестирования подтвердили эффективность подхода, продемонстрировав обнаружение и исправление уязвимостей в эталонных реализациях драйверов.
	
\end{abstractpage}

\tableofcontents

\section{Введение}
\subsection{Фаззинг}
Фаззинг (fuzzing) - это автоматизированный метод тестирования программного обеспечения, при котором система генерирует случайные или модифицированные входные данные с целью выявления ошибок, уязвимостей и сбоев в работе программы \cite{Kuliamin}. Данный метод тестирования позволяет находить ситуации, которые могут привести к неожиданному поведению программы или быть использованы злоумышленниками для эксплуатации уязвимостей \cite{Manes}.

Фаззинг стал ключевым инструментом обеспечения безопасности программного обеспечения благодаря высокой степени автоматизации и способности находить редкие и критические ошибки \cite{Manes}. Он играет важную роль в современной практике разработки ПО, особенно в условиях роста зависимости от open-source решений \cite{Boehme}.

Крупные компании (например, Google, Microsoft, Samsung) активно применяют фаззинг для поиска уязвимостей в своих продуктах. Таким обращом, фаззинг уже стал промышленным стандартом обеспечения качества и безопасности ПО \cite{Boehme}.

Существует несколько подходов к классификации фаззеров \cite{Kuliamin}\cite{Boehme}:
\begin{itemize}
	\item \textbf{По уровню доступа}:
	\begin{itemize}
		\item Черный ящик: фаззинг без использования информации о внутреннем состоянии программы.
		\item Серый ящик: использование частичной информации о покрытии кода.
		\item Белый яшик: полное знание структуры программы, часто используется символьное исполнение.
	\end{itemize}
	\item \textbf{По типу генерации входных данных}
	\begin{itemize}
		\item Мутационные фаззеры - изменяют существующие примеры входных данных.
		\item Генеративные фаззеры - создают новые данные <<по шаблону>>.
	\end{itemize}
	\item \textbf{По анализу выполнения:}
	\begin{itemize}
		\item На основе покрытия кода (coverage-guided fuzzing).
		\item Символьное исполнение (symbolic execution).
		\item Тайнт-анализ (Dynamic Taint Analysis, DTA) - метод динамического анализа программного обеспечения, предназначенный для отслеживания распространения <<помеченных>> данных в ходе выполнения программы. Позволяет, например, отследить утечку конфиденциальных данных \cite{Kuliamin}.
	\end{itemize}
\end{itemize}

Современные фаззеры используют различные техники для увеличения эффективности тестирования \cite{Kuliamin}\cite{Boehme}:
\begin{itemize}
	\item \textbf{Фаззинг с обратной связью по покрытию (Coverage-guided fuzzing)} - наиболее распространенная техника, при которой фаззер отслеживает участки кода, достигаемые текущим корпусом, и генерирует новые входные данные, которые открывают новые ветви выполнения.
	\item \textbf{Гибридный фаззинг}  комбинирует фаззинг с динамической символьной интерпретацией (DSE), позволяя целенаправленно исследовать сложные ветви логики.
	\item \textbf{Эволюционные алгоритмы} используются для оптимизации процесса мутации входных данных.
	\item \textbf{Модельный фаззинг (Model-based fuzzing)} полагается на формальные спецификации или прототипы формата для генерации корректных, но потенциально опасных входных данных.
\end{itemize}

Несмотря на широкое применение, фаззинг-тестирование имеет ряд ограничений \cite{Kuliamin}\cite{Boehme}:
\begin{enumerate}
	\item Труднорешаемые логические условия препятствуют продвижению фаззера.
	\item Ошибки зависят от состояния системы, что делает их сложно воспроизводимыми.
	\item Недерминированное поведение параллельных потоков усложняет анализ.
	\item Злоумышленники и разработчики ПО могут использовать обфускацию кода, маскировку ошибок и обнаружение режима фаззинга для снижения его эффективности.
\end{enumerate}

Таким образом, фаззинг остаётся одним из наиболее перспективных и действенных методов обеспечения надёжности и безопасности программного обеспечения в условиях растущего числа угроз и сложности современных систем.

\subsection{libFuzzer}
Тестовые сценарии разработаны с ипользованием библиотеки libFuzzer. \textbf{libFuzzer} - это встраиваемый фаззер с обратной связью по покрытию и поддержкой эволюционных алгоритмов мутации входных данных. Он разработан как часть проекта LLVM \cite{Libfuzzer}.

libFuzzer использует следующие техники и стратегии фаззинга:
\begin{itemize}
	\item \textbf{Фаззинг с обратной связью по покрытию}.
	\item \textbf{Эволюционные алгортмы} для мутации входных данных.
	\item \textbf{Поддержка словарей}: позволяет задавать характерные последовательности для формата входных данных.
	\item \textbf{Параллелизм}: поддерживает запуск нескольких процессов фаззинга с общим корпусом.
\end{itemize}

Ключевые компоненты libFuzzer:
\begin{itemize}
	\item \textbf{fuzz target}: точка входа для фаззера (целевая функция). Это пользовательская функция \FunName{LLVMFuzzerTestOneInput}, которая принимает массив байтов и передает его тестируемому коду. Эта функция должна быть детерминированной, для воспроизводимости тестов.
	\item \textbf{Корпус (Corpus)}: набор входных данных для фаззинга.
	\item \textbf{SanitizerCoverage}: инструмент LLVM, обеспечивающий сбор информации о покрытии кода. Он позволяет определить фаззеру, какие части кода были выполнены.
	\item \textbf{Механизмы мутации}: встроенные стратегии изменения входных данных, например, вставка, удаление, замена, скрещивание и другие.
	\item \textbf{Санитайзеры}: поддерживает интеграцию с AddressSanitizer, UBSan, MSan и LeakSanitizer.
\end{itemize}
\begin{figure}[htbp]
	\centering % Центрирование
	\includegraphics[width=0.8\textwidth]{Piclibfuzz.pdf} % Путь к файлу
	\caption{Общая схема работы фаззера с использованием libFuzzer.} % Подпись
	\label{intro:libfuzz} % Метка для ссылок
\end{figure}
Работа фаззера (см. рисунок \ref{intro:libfuzz}) начинается с инициализации корпуса, после чего запускается основной цикл: выбирается элемент из корпуса, мутирует, передается в целевую функцию и оценивается результат. В случае успешного увеличения покрытия или обнаружения ошибки (завершение с ошибкой, утечка памяти, таймаут) состояние сохраняется. Процесс продолжается бесконечно или прерывается в случае обнаружения ошибки. В последнем случае тест, вызвавший ошибку, сохраняется отдельно для последующего точечного воспроизведения.

\subsection{UEFI-окружение и драйверы}
\textbf{Unified Extensible Firmware Interface (UEFI)} - это интерфейс между операционной системой и прошивкой платформы (firmware), заменяющий устаревший BIOS.  Интерфейс обеспечивает взаимодействие между аппаратными средствами платформы и программным обеспечением, таким как загрузчик операционной системы. UEFI является ключевым компонентом современной архитектуры загрузки и инициализации аппаратных платформ, предоставляя расширенные возможности по сравнению с традиционной BIOS: поддержка модульности, драйверов, файловых систем, SecureBoot, сетевые функции и другие механизмы \cite{UEFISpec}.

В контексте UEFI драйвер представляет собой модуль, реализующий поддержку конкретного оборудования или логического устройства через один или несколько стандартизированных протоколов. Драйверы могут загружаться как вместе с прошивкой, так и с внешнего носителя по запросу приложения и работать в UEFI-среде до передачи упрвления ОС.

\begin{figure}[htbp]
	\centering % Центрирование
	\includegraphics[width=0.5\textwidth]{piphases.png} % Путь к файлу
	\caption{Фазы инициализации платформы \cite{UEFIPI}.} % Подпись
	\label{intro:piphases} % Метка для ссылок
\end{figure}
Спецификация \cite{UEFIPI} описывает архитектуру инициализации платформы, разбитую на несколько фаз (см. рисунок \ref{intro:piphases}), каждая из которых имеет свои особенности и допустимые типы драйверов:
\begin{itemize}
	\item \textbf{PEI Drivers, PEIM}. Выполняются в фазе PEI. Отвечают за минимальную инициализацию оборудования, используют ограниченный набор сервисов: PEI Services и PEI Private Interfaces.
	\item \textbf{DXE Drivers}. Основные драйверы, запускающиеся в фазе DXE (Driver Execution Enviroment), обеспечивают поддержку UEFI Boot Services и Runtime Services. Драйверы файловых систем обычно относятся к этому типу.
	\item \textbf{SMM Drivers}. Драйверы, которые работают в режиме SMM (System Managment Mode) и предназначены для обработки аппаратных прерываний вне контекста ОС (например, драйверы термоконтроля).
	\item \textbf{UEFI Application Driver}. Некоторые UEFI-приложения могут запустить по требованию драйвер с внешнего носителя.
\end{itemize} 

В контексте UEFI протоколы предоставляют собой стандартизированные интерфейсы, через которые компоненты среды могут предоставлять и использовать определенные функциональные возможности. Протоколы реализуются как структуры данных (таблицы), содержащие указатели на функции, различные поля с данными и указатели на другие протоколы. Протоколы идентифицируются уникальным GUID (Globally Unique Identifier), что позволяет найти все реализации конкретного протокола в среде.

 Драйвер файловой системы должен реализовывать протокол \VarName{EFI\_FILE\_PROTOCOL}. Данный протокол предоставляет унифицированный интерфейс для работы с файловой системой: открытие файлов, чтение/запись, удаление, получение информации о файлах, чтение содержимого каталогов. В спецификации определены две версии:
 \begin{itemize}
 	\item \VarName{EFI\_FILE\_PROTOCOL\_REVISION} описывает только синхронные операции.
 	\item \VarName{EFI\_FILE\_PROTOCOL\_REVISION2} описывает синхронные и асинхронные операции.
 \end{itemize}
 
В данной работе для тестирования драйверов применяется базовое UEFI-окружение в пользовательском пространстве, предоставляемое проектом \GitName{Acidanthera/OpenCorePkg} \cite{OpenCorePkg}. Ключевые преимущества этого подхода перед тестированием в реальной среде UEFI:
\begin{itemize}
	\item Полный доступ к функционалу libFuzzer и LLVM Sanitizer's:
	\begin{itemize}
		\item Запуск нескольких процессов фаззинга  одновременно.
		\item Информативное описание результатов тестирования.
		\item Простой доступ к тестам, вызывающим ошибки.
	\end{itemize}
	\item Отладка стандартными средствами GDB.
	\item Изолированное тестирование конкретного драйвера.
\end{itemize}
 
 Тестируются драйверы FAT, ext4 (предоставляемые проектом \GitName{Acidanthera/audk} \cite{Audk}) и NTFS, предоставляемый проектом \cite{OpenCorePkg}.  
\section{Подготовка окружения}
Как уже упоминалось ранее, мы тестируем драйверы файловых систем в минималистичном UEFI-оркужении в пространстве пользователя от \cite{OpenCorePkg}. В это окружение необходимо добавить:
\begin{itemize}
	\item Эмуляцию дискового устройства: предоставить интерфейсы (\VarName{EFI\_BLOCK\_IO\_PROTOCOL}, \VarName{EFI\_DISK\_IO\_PROTOCOL}, \VarName{EFI\_DISK\_IO2\_PROTOCOL}), через которые драйвер взаимодействует с носителем
	
	\item Эмуляцию подсистемы событий UEFI: для тестирования драйверов, поддерживающих асинхронный функционал 
\end{itemize} 

\subsection{Эмулирование дискового устройства}
\begin{figure}[htbp]
	\centering % Центрирование
	\includegraphics[width=0.8\textwidth]{MemoryIo.pdf} % Путь к файлу
	\caption{Интерфейсы, предоставляемые модулем MemoryIO} % Подпись
	\label{env:pic:memoryio} % Метка для ссылок
\end{figure}

В реальной UEFI-среде драйверы файловых систем получают доступ к носителю через стандартизированные интерфейсы (см. рисунок \ref{env:pic:memoryio})
\begin{itemize}
	\item \VarName{EFI\_BLOCK\_IO\_PROTOCOL} предоставляет доступ к устройству как к блочному, содержит информацию о носителе через \VarName{EFI\_BLOCK\_IO\_MEDIA}	
	\item \VarName{EFI\_DISK\_IO\_PROTOCOL} обеспечивает синхронный доступ к устройству как к непрерывному потоку байтов
	
	\item \VarName{EFI\_DISK\_IO2\_PROTOCOL} расширяет \VarName{EFI\_DISK\_IO\_PROTOCOL}, добавляя поддержку асинхронных операций чтения/записи/сброса
\end{itemize}
полное описание можно найти в \cite{UEFISpec}.


Для эмуляции этих интерфейсов был разработан модуль\textbf{ MemoryIO}. Его основная задача  - <<создать>> в оперативной памяти виртуальное дисковое устройство на основе тестовых данных, передаваемых libFuzzer. Модуль возвращает структуру, содержащую полные реализации всех четырех требуемых интерфейсов

Реализация функций в MemoryIO
\begin{enumerate}
	\item Синхронные операции (\FunName{ReadDisk}, \FunName{WriteDisk}) эмулируются простым копированием данных из/в буфер, соответствующий виртуальному диску в памяти. Функции \FunName{FlushBlocks} и \FunName{Reset} реализованы как заглушки, так как физического устройства не существует
	\item Асинхронные операции  (\FunName{ReadDiskEx}, \FunName{WriteDiskEx}, \FunName{FlushDiskEx})  эмулируют асинхронный ввод-вывод с использованием механизма событий UEFI:
	\begin{itemize}
		\item При вызове создается контекст задачи, содержащий тип операции, смещение, буфер данных и его размер
		\item Создается  сигнальное событие \VarName{EFI\_EVENT}, с которым ассоциируется обработчик задачи и ее контекст
		\item Создается однократный таймер EFI с небольшой задержкой (эмулирующей время выполнения операции). Этот таймер связывается с созданным событием
		\item По истечении времени таймера срабатывает событие, вызывается обработчик, который выполняет запланированную операцию, сигнализирует о завершении операции и освобождает ресурсы задачи
	\end{itemize}
	\item Функция \FunName{Cancel}  позволяет отменить все ожидающие асинхронные задачи
    \item Блочные функции (\FunName{ReadBlocks},\FunName{WriteBlocks}) не реализовывались, так как тестируемые драйверы не используют их напрямую
\end{enumerate}

После инициализации MemoryIO возвращаемые им интерфейсы передаются тестируемому драйверу файловой системы. Драйвер использует их для монтирования виртуального диска и предоставления дескриптора корневого каталога \VarName{EFI\_FILE\_PROTOCOL}, через который осуществляются все операции с файловой системой

\subsection{Эмуляция подсистемы событий UEFI Event}

Для обеспечения работы асинхронных операций и таймеров, задействованных как драйверами файловых систем, так и модулем MemoryIO, был реализован модуль \textbf{YummyEvent}. Он предоставляет программную эмуляцию ключевых функций подсистемы событий UEFI Boot Services \cite{UEFISpec}, включая:
\begin{itemize}
	 \item Создание/уничтожение событий \FunName{CreateEvent}/\FunName{CloseEvent}
	 \item Сигнализация событий \FunName{SignalEvent}
	 \item Ожидание событий \FunName{WaitForEvent}
	 \item Управление таймерами \FunName{SetTimer}
	 \item Контроль уровня привелегий \FunName{RaiseTPL}/\FunName{RestoreTPL}
	 \item Проверка состояния событий \FunName{CheckEvent}
	 \item Поддерживаемые типы событий: сигнальные, ожидающие, периодические таймеры
\end{itemize}

Особенность реализации заключается в принципе явной диспетчеризации событий. Обработка событий (включая срабатывание таймеров и вызов callback-функций) не происходит в фоновых потоках. Вместо этого она инициируется исключительно при вызове методов UEFI Event, например таких как \FunName{SignalEvent}, \FunName{WaitForEvent} или \FunName{RaiseTPL}. Такой подход обеспечивает следующие характеристики:
\begin{itemize}
	\item Обработчики событий выполняются в контексте вполняющегося кода, что эмулирует поведение прерываний в UEFI среде
	\item Отсутствие фоновой асинхронности гарантирует воспроизводимость результатов фаззинга
	\item Упрощение отладки
\end{itemize}

Для решения проблемы своевременного срабатывания таймеров введена функция \FunName{YummyEventsDispatch}. Её принудительный вызов позволяет <<продвинуть>> системное время, сработать ожидающим таймерам и обработать накопившиеся события. 

\section{Метод тестирования}

Разработанный метод фазинг-тестирования реализует комплексный подход, сочетающий синхронные и асинхронные сценарии для максимального покрытия функционала. Основу методики составляют шесть взаимодополняющих алгоритмов.

\textbf{FsFuzzSubTest1} выполняет работу с отдельным файлом: открытие на чтение, чтение метаданных файла, последовательное чтение блоков данных с перемещением позиции, а затем открытие на запись для записи данных в конец файла. Алгоритм выявляет ошибки управления файловыми дескрипторами, обработки позиционирования и операций ввода-вывода. Обобщенная схема работы алгоритма представлена на рисунке \ref{met:pic:fsfuzzsubtesti}.
\begin{figure}[htbp]
	\centering % Центрирование
	\includegraphics[width=0.5\textwidth]{FsFuzzSubtestI.pdf} % Путь к файлу
	\caption{Обобщенный алгоритм обработки файла \FunName{FsFuzzSubtest1}.} % Подпись
	\label{met:pic:fsfuzzsubtesti} % Метка для ссылок
\end{figure}

\newpage
\textbf{FsFuzzTest1} реализует рекурсивный обход файловой системы с заданной глубиной. Для каждого обнаруженного файла он запускает \FunName{FsFuzzSubtest1}, а для катологов - рекурсивно углубляется в иерархию. Обобщенная схема алгоритма представлена на рисунке \ref{met:pic:fsfuzztesti}.
\begin{figure}[htbp]
	\centering % Центрирование
	\includegraphics[width=0.7\textwidth]{FsFuzzTestI.pdf} % Путь к файлу
	\caption{Обобщенный алгоритм обхода каталога\FunName{FsFuzzTest1}.} % Подпись
	\label{met:pic:fsfuzztesti} % Метка для ссылок
\end{figure}

\textbf{FsFuzzTest2} последовательно открывает и удаляет все обычные файлы в текущем каталоге. Обобщенная схема алгоритма представлена на рисунке \ref{met:pic:fsfuzztestii}. 
\begin{figure}[htbp]
	\centering % Центрирование
	\includegraphics[width=0.7\textwidth]{FsFuzzTestII.pdf} % Путь к файлу
	\caption{Обобщенный алгоритм очистки каталога\FunName{FsFuzzTest2}.} % Подпись
	\label{met:pic:fsfuzztestii} % Метка для ссылок
\end{figure}

\textbf{FsFuzzTest3} создает цепочку вложенных директорий и файл на последнем уровне вложенности. Важно, что алгоритм создает файл, используя длинный путь к нему, но для успешного завершения этой операции необходимо создать все подпапки последовательно. Обобщенная схема алгоритма представлена на рисунке \ref{met:pic:fsfuzztestiii}.
\begin{figure}[h]
	\centering % Центрирование
	\includegraphics[width=0.7\textwidth]{FsFuzzTestIII.pdf} % Путь к файлу
	\caption{Обобщенный алгоритм создания тестового файла на длинном пути \FunName{FsFuzzTest3}.} % Подпись
	\label{met:pic:fsfuzztestiii} % Метка для ссылок
\end{figure}

\textbf{FsFuzzAsyncWorker} - обработчик результата выполнения асинхронной задачи. После успешного открытия файла в асинхронном режиме он планирует асинхронное чтение из этого файла. После успешного  чтения он планирует асинхронную запись. После успешной записи \FunName{FsFuzzAsyncWorker} помечает асинхронную операцию выполненной, чтобы в дальнейшем освободить все занятые  задачей реурсы. Обобщенная схема алгоритма представлена на рисунке \ref{met:pic:fsfuzzasyncworker}.
\begin{figure}[h]
	\centering % Центрирование
	\includegraphics[width=0.7\textwidth]{FsFyzzAsyncWorker.pdf} % Путь к файлу
	\caption{Обобщенный алгоритм обработки асинхронной задачи \FunName{FsFuzzAsyncWorker}.} % Подпись
	\label{met:pic:fsfuzzasyncworker} % Метка для ссылок
\end{figure}

\newpage
\textbf{FsFuzzTest4} дополняет асинхронную проверку, планируя параллельное выполнение операций \FunName{FsFuzzAsyncWorker} для всех файлов в каталоге. Обобщенная схема алгоритма представлена на рисунке \ref{met:pic:fsfuzztestiv}.
\begin{figure}[htbp]
	\centering % Центрирование
	\includegraphics[width=0.7\textwidth]{FsFuzzIV.pdf} % Путь к файлу
	\caption{Обобщенный алгоритм планирования асинхронных операций над файлами из каталога \FunName{FsFuzzTest4}.} % Подпись
	\label{met:pic:fsfuzztestiv} % Метка для ссылок
\end{figure}

Описанные сценарии тестирования используют стандартизированные функции из \VarName{EFI\_FILE\_PROTOCOL} и не вызывают каких-то узкоспециализированные функции драйвера напрямую. Это позволяет применять их для любых драйверов файловых систем без дополнительных правок. Все сценарии последовательно вызываются из главной функции фаззинг-тестирования \FunName{FsFuzzTest}, которая принимает на вход только дескриптор корневого каталога. Алгоритм функции представлен на рисунке \ref{met:pic:fsfuzztest}.
\begin{figure}[htbp]
	\centering % Центрирование
	\includegraphics[width=0.75\textwidth]{FsFuzzTest.pdf} % Путь к файлу
	\caption{Главная функция, прводящая фаззинг-тестирование \FunName{FsFuzzTest}.} % Подпись
	\label{met:pic:fsfuzztest} % Метка для ссылок
\end{figure} 

Все перечисленные функции поставляются модулем \textbf{FsFuzzTest}. В завершении на рисунке \ref{met:pic:testscheme} приведена обобщенная схема тестирования драйверов файловой системы, предложенная в этой работе.
\begin{figure}[htbp]
	\centering % Центрирование
	\includegraphics[width=0.8\textwidth]{TestScheme.pdf} % Путь к файлу
	\caption{Полная обобщенная схема предлагаемого метода фаззинг-тестирования.} % Подпись
	\label{met:pic:testscheme} % Метка для ссылок
\end{figure} 
\section{Результаты тестирования драйвера FAT}

\subsection{Подготовка к тестированию}
Для запуска фаззинг-тестирования драйвера файловой системы FAT необходим начальный корпус из тестовых образов. Их генерация автоматизирована с помощью набора bash-скриптов (\textit{Test1.sh}-\textit{Test7.sh}). Скрипты создают образы, моделирующие разнообразные сценарии использования FAT.  Общая характеристика генерируемых образов:
\begin{itemize}
	\item Создаются образы FAT12, FAT16 и FAT32.
	\item Различные размеры образов: от 8 Мб (маленькие образы для FAT12/FAT16) до 64 Мб (стандартный размер для FAT32).
	\item Используются разные размеры кластеров (512 байт и 2 Кб).
	\item Создается заданное количество папок (от 1 до 5) с файлами фиксированного (8 Кб) размера.
	\item В корне размещаются файлы значительно большого размера (512 Кб, 2 Мб).
	\item Создаются цепочки вложенных папок (глубина от 2 до 3 уровней) с файлами на конечном уровне вложенности.
	\item Большинство скриптов создает специальную папку FILLSPACE, которая заполняется по остаточному принципу до достижения заданного порога свободного места (1 Мб, 2 Мб, 8Мб) для имитации работы с почти заполненными образами.
	\item Имена файлов и папок (длиной 32 символа) генерируются случайно из буквенно-цифрового набора. Скрипт \textit{Test7.sh} дополнительно включает имена с пробелами и специальными символами, проверяя корректность их обработки.
	\item Создаваемые файлы заполняются случайными данными из Linux-устройства \textit{/dev/urandom}.
\end{itemize} 
Такой подход к формированию начального корпуса позволяет эффективно тестировать базовые функции драйвера (чтение, запись, поиск, открытие/закрытие файлов и каталогов) в экстремальных и разнообразных условиях. Подробные параметры каждого скрипта сведены в таблице \ref{fat:tab:test_detail}.
\begin{table}[htbp]
	\renewcommand{\arraystretch}{1.5}
	\centering
	\begin{tabular}{|c|c|c|p{10cm}|}
		\hline
		\textbf{Скрипт} & \textbf{Размер} & \textbf{FAT} & \textbf{Описание} \\
		\hline
		\textit{Test1.sh} & 64 Мб & 32 &   4 корневые папки (по 10 файлов 8 Кб), 3 вложенные цепочки (глубина 3) с 10 файлами по 8 Кб на конце, 5 файлов в корне по  2Мб. FILLSPACE до 8 Мб. Размер кластера 512 байт. \\
		\hline
		\textit{Test2.sh} & 32 Мб & 16 &   Структура идентична Test1. \\
		\hline
		\textit{Test3.sh} & 64 Мб & 16 &  Структура идентична Test2, но размер кластера увеличен до 2 Кб.\\
		\hline
		\textit{Test4.sh} & 8 Мб & 12 &  2 корневые папки (по 5 файлов 8Кб), 1 вложенная цепочка (глубина 3) с 5 файлами по 8 Кб на конце,  3 файла в корне по 4 Кб. FILLSPACE до 2 Мб. Размер кластера 2 Кб. \\
		\hline
		\textit{Test5.sh} & 64 Мб & 32 &  5 корневых папок (по 15 файлов 8 Кб), 4 вложенные цепочки (глубина 3) с 15 файлами по 8 Кб на конце, 2 файла  в корне по 2 МБ. Без заполнения FILLSPACE.  Размер кластера 2 Кб.\\
		\hline
		\textit{Test6.sh} & 8 Мб & 16 &  Минималистичная структура: 1 корневая папка (3 файла), 1 вложенная цепочка (глубина 2, 3 файла), 2 файла по 512 КБ.  FILLSPACE до 1 Мб. Размер кластера 512 байт.\\
		\hline
		\textit{Test7.sh} & 64 Мб & 32 & 3 корневые папки (по 5 файлов), 4 файла по 1 МБ, 2 вложенные цепочки (глубина 3, 3 файла). Имена с пробелами и спецсимволами. Без FILLSPACE. Размер кластера 512 байт.  \\
		\hline
		
	\end{tabular}
	\caption{Подробные характеристики скриптов для создания образов.}
	\label{fat:tab:test_detail}
\end{table}

Для тестирования асинхронного функционала драйвера FAT было отключено кеширование при вызове асинхронных функций, чтобы не ждать переполнение кеша, иначе операции все равно будут выполнены синхронно. Для этого в файле \FName{ReadWrite.c} драйвера были исправлены функции \FunName{FatReadEx} и \FunName{FatWriteEx} (см. листинг \ref{fat:listingA}): при вызове функции \FunName{FatIFileAccess} передаём флаги \FlagName{ReadDisk}/\FlagName{WriteDisk} вместо \FlagName{ReadData}/\FlagName{WriteData}.
\lstinputlisting[
caption={Файл \FName{ReadWrite.c}: изменения в функциях \FunName{FatReadEx} и \FunName{FatWriteEx}.},
label={fat:listingA}
]{fatlistings/listingA.c} 

\subsection{Исправленные ошибки}

В процессе фаззинг-тестирования были обнаружены и исправлены следующие критические ошибки:
\begin{enumerate}
	\item \textbf{Неопределенное поведение (UB)} в функциях \FunName{FatGetDirEntInfo} и \FunName{FatOpenDirEnt}.
	
	\textbf{Файл:} \FName{DirectoryManage.c}.
	
	\textbf{Проблема:} При сборке номера кластера \VarName{Cluster} из 16-битных компонент сдвиг влево на 16 бит вызвал неявное приведение к типу \TypeName{int}, что привело к \ErrorName{UB} при переполнении \cite{ISO_C17}.
	
	\textbf{Исправление:} Явное приведение к \TypeName{UINTN} (см. листинги \ref{fat:listingB} и \ref{fat:listingC}).
	\lstinputlisting[
	caption={Файл \FName{DirectoryManage.c}, кусок кода функции \FunName{FatGetDirEntInfo}:  исправление ошибки \ErrorName{UB}.},
	label={fat:listingB}
	]{fatlistings/listingB.c} 
	\lstinputlisting[
	caption={Файл \FName{DirectoryManage.c}, кусок кода функции \FunName{FatOpenDirEnt}:  исправление ошибки \ErrorName{UB}.},
	label={fat:listingC}
	]{fatlistings/listingC.c} 
	
	\item \textbf{Утечка памяти (MemoryLeak) и взаимоблокировка (DeadLock)} в функции \FunName{FatIFileAccess}.
	
	\textbf{Файл:} \FName{ReadWrite.c}.
	
	\textbf{MemoryLeak:} При ошибке в \FunName{FatGrowEof} не освобождается объект асинхронной задачи.
	
	\textbf{DeadLock:} Блокировка тома не снималась перед вызовом \FunName{FatQueueTask}, что может вызвать взаимоблокировку при обработке связанного с задачей сигнала.
	
	\textbf{Исправление:}
	\begin{itemize}
		\item Добавлено освобождение задачи при ошибках (строки 9-11 листинга \ref{fat:listingD}).
		\item Снятие блокировки тома перед \FunName{FatQueueTask} и при ошибках (строки 13 и 19 листинга  \ref{fat:listingD}).
	\end{itemize}
	\lstinputlisting[
	caption={Файл \FName{ReadWrite.c}, кусок кода функции \FunName{FatIFileAccess}:  исправление ошибок \ErrorName{MemoryLeak} и \ErrorName{DeadLock}.},
	label={fat:listingD}
	]{fatlistings/listingD.c} 
	
	\item \textbf{Доступ после освобождения (UseAfterFree)} в \FunName{FatQueueTask}.
	
	\textbf{Файл:} \FName{Misc.c}.
	
	\textbf{Проблема:} Итерация по списку подзадач без блокировок приводила к доступу к освобождённой памяти, если элементы удалялись в обработчике событий.
	
	\textbf{Исправление:} Добавлены блокировки \VarName{FatTaskLock} во время итераций по списку (см. листинг \ref{fat:listingE}).
	\lstinputlisting[
	caption={Файл \FName{Misc.c}:  переписанная функция \FunName{FatQueueTask}, многоточиями отмечены похожие с предыдущей версией места.},
	label={fat:listingE}
	]{fatlistings/listingE.c} 
	
\end{enumerate}

Итоговое количество обнаруженных и исправленных на данный момент ошибок сведено в таблице \ref{fat:tab:sumerror}.
\begin{table}[htbp]
	\renewcommand{\arraystretch}{1.5}
	\centering
	\begin{tabular}{|c|c|}
		\hline
		\textbf{Тип} & \textbf{Количество} \\
		\hline
		\ErrorName{UB} & 2 \\
		\hline
		\ErrorName{MemoryLeak} & 1 \\
		\hline
		\ErrorName{DeadLock} & 1 \\
		\hline
		\ErrorName{UseAfterFree} & 1 \\
		\hline
	\end{tabular}
	\caption{Тип и количество исправленных ошибок в драйвере FAT.}
	\label{fat:tab:sumerror}
\end{table}

\newpage
\subsection{Покрытие кода}
Результирующее покрытие кода драйвера FAT сведено в таблице \ref{fat:tab:coverage}.
\begin{table}[ht]
	\centering
	\begin{tabular}{|c|c|c|c|}
		\hline
		\textbf{Filename} & \textbf{Line} & \textbf{Branch} & \textbf{Function} \\
		\hline
		\FName{Delete.c}&  58.3\% &  32.4\% &  100.0\% \\
		\FName{DirectoryCache.c} &  100.0\% &  70.6\% &  100.0\% \\
		\FName{DirectoryManage.c} &  76.7\% &  62.9\% &  80.8\% \\
		\FName{DiskCache.c} &  80.1\% &  76.6\% &  85.7\% \\
		\FName{FileName.c} &  89.1\% &  86.7\% &  100.0\% \\
		\FName{FileSpace.c} &  88.0\% &  79.0\% &  91.7\% \\
		\FName{Flush.c} &  86.6\% &  70.2\% &  100.0\% \\
		\FName{Hash.c} &  100.0\% &  100.0\% &  100.0\% \\
		\FName{Info.c} &  37.6\% &  30.4\% &  55.6\% \\
		\FName{Init.c} &  73.3\% &  51.1\% &  66.7\% \\
		\FName{Misc.c} &  87.2\% &  62.8\% &  94.1\% \\
		\FName{Open.c} &  85.4\% &  72.9\% &  100.0\% \\
		\FName{OpenVolume.c} &  84.6\% &  50.0\% &  100.0\% \\
		\FName{ReadWrite.c} &  87.3\% &  69.4\% &  100.0\% \\
		\FName{UnicodeCollation.c} &  47.4\% &  0.0\% &  71.4\% \\
		\hline
	\end{tabular}
	\caption{Покрытие кода драйвера FAT.}
	\label{fat:tab:coverage}
\end{table}
\subsection{Результаты тестирования ext4}

Тут тоже какой-то текст
\section{Результаты тестирования драйвера EFI NTFS}

\subsection{Подготовка к тестированию}
Аналогично FAT и ext4 был разработан набор bash-скриптов (\textit{Test1.sh}-\textit{Test8.sh}), создающих специализированные тестовые образы файловой системы NTFS, которые составят начальный корпус для фаззинг-тестирования. Эти скрипты охватывают ключевые особенности NTFS, включая поддержку различных размеров кластеров, работу с метаданными, обработку сложных структур ссылок и экстремальных случаев. Подробные характеристики приведены в таблице \ref{ntfs:tab:test_detail}. Основные характеристики генерируемых образов:
\begin{itemize}
	\item Многоуровневые вложенные директории (до 4 уровней).
	\item Различные типы ссылок (символические и жесткие).
	\item Специальные случаи (ссылки на себя, корневые директории, за пределы образа).
	\item Различные размеры кластеров.
	\item Различные размеры образов (8 Мб - 32 Мб).
	\item Аналогичное заполнение до указанного порога (FILLSPACE, только Test1.sh).
	\item Генерация как стандартных, так и длинных имен со специальными символами.
	\item Жесткие ссылки на один файл.
\end{itemize} 

\begin{table}[htbp]
	\renewcommand{\arraystretch}{1.5}
	\centering
	\begin{tabular}{|c|c|p{10cm}|}
		\hline
		\textbf{Скрипт} & \textbf{Размер} & \textbf{Описание} \\
		\hline
		\textit{Test1.sh} & 16 Мб & Базовая структура: 4 корневые папки (по 10 файлов), 3 вложенные цепочки (глубина 3), 3 файла по 2 Мб в корне. FILLSPACE до 2 Мб. \\
		\hline
		\textit{Test2.sh} & 32 Мб & Нестандартный размер кластера (8 Кб). 5 корневых папок (по 15 файлов), 4 вложенные цепочки (глубина 4), 10 файлов по 1 Мб.\\
		\hline
		\textit{Test3.sh} & 32 Мб &Длинные имена (200 символов) со спецсимволами. 20 папок (по 200 файлов по 1 Кб) + символические ссылки на себя и из корня. \\
		\hline
		\textit{Test4.sh} & 8 Мб & Маленький образ. 20 папок (по 10 файлов по 1 Кб) + символические ссылки на себя в каждой папке и из корня на папки.\\
		\hline
		\textit{Test5.sh} & 16 Мб & Жесткие ссылки: 1 базовый файл (2 Мб) + 10 ссылок в корне + 5 ссылок в подпапке.\\
		\hline
		\textit{Test6.sh} & 16 Мб & Массовое создание файлов: 10 папок × 500 файлов (всего 5000). \\
		\hline
		\textit{Test7.sh} & 16 Мб & Папки с символическими ссылками: 5 папок (по 10 файлов), в каждой символические ссылки на себя и на корневую директорию.\\
		\hline
		\textit{Test8.sh} & 16 Мб & Символическая ссылка на внешнюю директорию: 5 папок (по 10 файлов), в каждой символические ссылки на себя, корень и внешнюю папку за пределами образа.\\
		\hline
	\end{tabular}
		\caption{Подробные характеристики скриптов для создания образов.}
	\label{ntfs:tab:test_detail}
\end{table}

\newpage
\subsection{Ошибки, обнаруженные при тестировании}
\begin{enumerate}
	\item \textbf{Не следование спецификации.}
	
	\textbf{Файл:} \FName{NTFS.c}.
	
	\textbf{Проблема:} В функции \FunName{NTFSStart} указывается некорректная версия файлового протокола (см. листинг \ref{ntfs:listingA}), вторая версия файлового протокола требует наличия асинхронных функций, которые драйвер не поставляет \cite{UEFISpec}.
    
    \textbf{Решение:} Указать правильный номер версии.
	\lstinputlisting[
	caption={Файл \FName{NTFS.c}, кусок кода функции \FunName{NTFSStart}: некорректная версия файлового протокола},
	label={ntfs:listingA}
	]{ntfslistings/listingA.c} 
	
	\item \textbf{Не возвращается признак конца каталога}.
	
	\textbf{Файл:} \FName{Open.c}.
	
	\textbf{Проблема:} Функция \FunName{FileRead} не возвращает признак конца каталога, как того требует спецификация \cite{UEFISpec}.
	
	\textbf{Текущий статус:} Не решена.
\end{enumerate}

\subsection{Покрытие кода}
На момент сдачи работы измерение покрытия кода для драйвера EFI NTFS \textbf{невозможно} из-за критической ошибки, приводящей к зависанию тестов при работе с каталогами. Для получения валиндных метрик необходимо исправить эту ошибку. 
\section{Заключение}
В ходе выполнения курсовой работы разработан метод фаззинг-тестирования драйверов файловых систем в UEFI-окружении. Основные результаты включают:
\begin{enumerate}
	\item Доработку пользовательского окружения подсистемой событий UEFI Event.
	\item Реализацию шести тестовых сценариев, охватывающих синхронные и асинхронные операции с файлами и каталогами через стандартизированный интерфейс \VarName{EFI\_FILE\_PROTOCOL}.
	\item Автоматизированную генерацию начальных корпусов из образов тестируемых файловых систем.
	\item Обнаружение и исправление ошибок в драйверах.
	\item Получение метрик покрытия кода для драйверов FAT и ext4.
\end{enumerate}

В продолжении этой работы планируется:
\begin{itemize}
	\item Продолжение фаззинг-тестирования драйверов FAT, ext4 и NTFS с исправлением выявленных ошибок.
	\item Оптимизация используемых сценариев, подбор оптимальных параметров, чтобы ускорить тестирование.
	\item Исследовать сценарий, при котором синхронные операции не дожидаются завершения асинхронных.
	\item Исследование ресурсоемкости метода. Даже несмотря на использование различных типов аллокаторов экспоненциально растет потребление памяти, что не дает запустить фаззер на долгое время. Возможно требуется провести исследование потребления памяти у используемого окружения.
\end{itemize} 

% СПИСОК ЛИТЕРАТУРЫ
\bibliographystyle{gost71u}
\bibliography{references}

\end{document}